\section{Příklad č.~4}

\subsection{Zadání}

Pro napájecí napětí platí: $u_{1} = U_{1} \cdot \sin (2 \pi ft)$, $u_{2} = U_{2} \cdot \sin (2 \pi f t)$. \\
Ve vztahu pro napětí $u_{C_{2}} = U_{C_{2}} \cdot \sin (2 \pi ft + \phi_{C_{2}})$ určete $|U_{C_{2}}|$ a $\varphi_{C_{2}}$. \\
Použijte metodu smyčkových proudů. \\
Pomocné směry šipek napájecích zprojů platí pro speciální časový okamžik $(t = \frac{\pi}{2\omega})$. \\

\begin{table}[ht]
	\centering
	\resizebox{\textwidth}{!}{
	\begin{tabular}{|c|c|c|c|c|c|c|c|c|c|c|}
		\hline
		sk. & $U_{1}$~[V] & $U_{2}$~[V] & $R_{1}~[\Omega]$ & $R_{2}~[\Omega]$ & $R_{3}~[\Omega]$ & $L_{1}$~[mH] & $L_{2}$~[mH] & $C_{1}$~[$\mu$F] & $C_{2}$~[$\mu$F] & $f$~[Hz] \\
		\hline
		& & & & & & & & & &  \\
		\hline
	\end{tabular}
	}
\end{table}

\begin{figure}[!h]
	\centering
	\begin{circuitikz}
		\draw (0, 0) to[sV=$u_{1}$] (0, -2) -- (4, -2) to[R=$R_{3}$] (6, -2) -- (6, 0);
		\draw (0, 0) -- (0, 2) to[C=$C_{1}$] (4, 2) to[R=$R_{1}$] (6, 2) to[sV=$u_{2}$] (6, 0);
		\draw (0, 0) to[L=$L_{1}$, *-] (2, 0) to[R=$R_{2}$] (4, 0) to[L=$L_{2}$, -*] (6, 0);
		\draw (4, 0) to[C=$C_{2}$, v=$u_{C_{2}}$, i=$i_{C_{2}}$, *-*] (4, -2);
	\end{circuitikz}
\end{figure}

\subsection{Řešení}

\subsection{Výsledek}
