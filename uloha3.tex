\section{Příklad č.~3}

\subsection{Zadání}

Stanovte napětí $U_{R_{3}}$ a proud $I_{R_{3}}$.
Použijte metodu uzlových napětí ($U_{A}$, $U_{B}$, $U_{C}$). \\

\begin{table}[ht]
	\centering
	\begin{tabular}{|c|c|c|c|c|c|c|c|c|}
		\hline
		sk. & $U$~[V] & $I_{1}$~[A] & $I_{2}$~[A] & $R_{1}~[\Omega]$ & $R_{2}~[\Omega]$ & $R_{3}~[\Omega]$ & $R_{4}~[\Omega]$ & $R_{5}~[\Omega]$ \\
		\hline
		& & & & & & & &  \\
		\hline
	\end{tabular}
\end{table}

\begin{figure}[!h]
	\centering
	\begin{circuitikz}
		\draw (0, -4) to[dcvsource=$U$] (0, -2) to[R=$R_{1}$] (0, 0) -- (2, 0) to[R=$R_{2}$, v=$U_{A}$, *-*] (2, -4) -- (0, -4);
		\draw (2, 0) -- (2, 2) to[ioosource=$I_{1}$] (6, 2) -- (6, 0) to[R=$R_{3}$, i=$I_{R_{3}}$, v=$U_{R_{3}}$, *-*] (2, 0);
		\draw (6, 0) to[R=$R_{4}$] (6, -4) -- (8, -4) to[ioosource=$I_{2}$] (8, 0) -- (6, 0);
		\draw (6, -4) to[R=$R_{5}$, v=$U_{C}$, *-*] (2, -4);
		\draw (6, 0) to[open, v=$U_{B}$] (2, -4);
	\end{circuitikz}
\end{figure}

\subsection{Řešení}

\subsection{Výsledek}
